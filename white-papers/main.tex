
\documentclass[11pt,reqno]{article}

\usepackage{amscd}
\usepackage{amsfonts}
\usepackage{amsmath}
\usepackage{amssymb}
\usepackage{amsthm}
\usepackage{fancyhdr}
\usepackage{latexsym}
\usepackage[colorlinks=true, pdfstartview=FitV, linkcolor=blue,
            citecolor=blue, urlcolor=blue]{hyperref}
\usepackage[pdftex]{graphicx}
\usepackage{epstopdf}

% Any custom macros or packages can go here
\newtheorem{remark}{Remark}

\title{\textbf{CSC-411 Assignment 1}}
\author{Anthony Ozog}
\date{\today}
\markboth{title}{A. Ozog}

\begin{document}

\maketitle
% I do not feel I needed an abstract
% \begin{abstract}
  % This is the abstract
% \end{abstract}

\section*{Introduction}
  Since I am a math major I have explored most programming and software development I know outside of my studies and as a result I have never gotten to implement any of these sorting algorithms before. I really took this as an opportunity to both learn these new algorithms and practice some of the tools I like using (i.e. LaTeX, rust, R, and Test Driven Development). As a bonus I also got to see the capabilities of the rust language, which I am quite fond of, and I would say the language/compiler preformed quite well, at first I was measuring in Microsecond and I was still getting 0 results for smaller 'n's' so I switched to measuring Nanoseconds 

\section{Sorting Implementations}
% Do not forget the brief note on minor optimizations/ and design choices in the rust code

\section{Input Generation}
% Do not forget to document how the inputs are made

\section{Timing}

\section{Analysis}

\section{Degraded Spatial Locality}


\end{document}
       
